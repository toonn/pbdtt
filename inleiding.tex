\chapter{Inleiding}
\label{inleiding}

In deze thesis worden twee programmeertalen met een sterk typesysteem
vergeleken. Dit met een nadruk op het typesysteem van de talen, eerder dan de
talen zelf. Het gaat hier om twee functionele programmeertalen: Agda
\cite{agda}, een dependently typed programmeertaal, en Haskell \cite{haskell},
een taal met een Hindley-Milner \cite{hindley} typesysteem.


\section{Typesystemen}

Een typesysteem is een verzameling van regels waaraan code in een bepaalde
programmeertaal moet voldoen. Er zijn verschillende redenen om zulke regels op
te leggen. De belangrijkste is type safety; dit betekent dat de computer nooit
onzinnige bewerkingen zal uitvoeren. Wat een onzinnige bewerking juist is,
hangt af van het typesysteem. Gewoonlijk worden bewerkingen zoals optelling
enkel gedefinieerd op getallen. Een voorbeeld van een onzinnige bewerking zou
dan een optelling kunnen zijn van een getal en een lijst. Een andere vorm van
bescherming die onder type safety valt is memory safety. Een belangrijke vorm
hiervan is bescherming tegen buffer overflows: deze hebben vaak security
vulnerabilities tot gevolg maar als het typesysteem over genoeg informatie
beschikt, kan gegarandeerd worden dat ze niet kunnen voorkomen.  Een andere
reden voor een typesysteem is bijvoorbeeld optimalisatie: als de compiler meer
informatie heeft kunnen er soms betere optimalisaties gedaan worden. Types
kunnen ook een vorm van documentatie zijn: een typesynoniem kan bijvoorbeeld
verduidelijken dat een functie een naam verwacht in plaats van gewoon een
string, wat het type is voor een opeenvolging van karakters.

Er is een grote variatie aan bestaande typesystemen. Het typesysteem voor de
programmeertaal C \cite{C} heeft eenvoudige types die in direct verband staan
met de voorstelling van de bijhorende waarden op het niveau van bits en bytes.
Het typesysteem van Java \cite{Java} daarentegen is volledig gericht op de
objectgeoriënteerde metafoor: een object wordt gedefinieerd door een klasse en
die klasse \emph{is} tevens het type voor dat object. Het Hindley-Milner
typesysteem wordt vaak gebruikt als basis voor het typesysteem van functionele
programmeertalen. Het meest merkwaardige kenmerk van dit typesysteem is dat het
meest algemene type voor een waarde altijd geïnfereert kan worden. Het
resultaat hiervan is dat een programma geschreven kan worden zonder enige type
annotatie en het typesysteem toch kan garanderen dat type safety behouden
blijft. Het typesysteem van Haskell is gebaseerd op het Hindley-Milner
typesysteem maar heeft een groot aantal uitbreidingen hierop. Zulke
uitbreidingen maken het typesysteem expressiever.


\section{Dependent Types}

Dependent types zijn nauw verbonden met de Curry-Howard correspondence
\cite{curryhow}. Ze breiden het verband tussen types en propositielogica uit tot
een predicatenlogica door het mogelijk te maken om de logische kwantoren voor
te stellen. Het verschil met andere typesystemen is dat een dependent type kan
afhangen van waarden, zo kan het type van een lijst afhangen van de lengte van
die lijst. Deze op het eerste zicht kleine verandering heeft een grote impact
op de expressiviteit van het typesysteem. Plotseling wordt het mogelijk om
ontzettend veel eigenschappen van de code op te nemen in de types waardoor ze
statisch gegarandeerd worden. Schendingen van de eigenschappen die we gecodeerd
hebben in de types zijn dus schendingen van de type safety en het typesysteem
laat zulke programma's niet toe.

De eigenschappen die we opnemen in de types gaan van heel eenvoudig,
bijvoorbeeld de lengte van een lijst, tot eerder ingewikkeld, bijvoorbeeld
invarianten van zoekbomen. Fundamenteel is er geen limiet, wat in de logica uit
te drukken is, is ook in de types uit te drukken. Echter als we ons bezig
houden met programmeren, eerder dan bewijzen formaliseren, zijn de
eigenschappen die we wensen af te dwingen vaak redelijk eenvoudig. We gebruiken
de types eerder om eigenschappen die we in talen zonder dependent types
impliciet laten - en dus in het achterhoofd moeten houden - expliciet te maken
en zo door het typesysteem te laten nakijken.

De typentheorie waarop Agda gebaseerd is, gelijkt op de intuïtionistische
typentheorie van Per Martin-Löf \cite{martinlof}. In deze thesis laten we dit
echter terzijde en bekijken we Agda als een programmeertaal met dependent types
in tegenstelling tot Haskell, wat als voorbeeld dient voor een programmeertaal
die in de praktijk toegepast wordt voor grootschalige systemen. Deze
veralgemening schiet natuurlijk tekort, Agda is geen perfecte voorstelling van
\emph{alle} programmeertalen met dependent types en Haskell is niet de enige
programmeertaal die op industriële schaal gebruikt kan worden met een sterk
typesysteem.


\section{Vergelijking}

De vergelijking tussen Agda en Haskell wordt gemaakt aan de hand van twee
concrete gevalstudies. Het gaat hier om twee probleemstellingen die eerst
geïmplementeerd zijn in Agda en daarna als het ware vertaald zijn naar Haskell.
Dit zorgt ervoor dat Agda als het ware bevoordeeld is om een elegante
implementatie te hebben maar dat is irrelevant omdat het niet de bedoeling is
een waardeoordeel te vellen. De gevalstudies dienen in de eerste plaats om te
illustreren wat we precies met dependent types kunnen uitdrukken dat anders
niet in de taal zelf uit te drukken is. De vergelijking tussen de twee talen is
er vooral om aan te tonen dat deze manier van programmeren ook in een courante
taal al mogelijk is. In hoofdstuk \ref{ch:agda-haskell} overlopen we de
features die we gebruiken uit Agda en Haskell.
