\chapter{Inleiding}
\label{inleiding}

types
dependent types
agda
haskell
vec relalg

In deze thesis worden twee programmeertalen met een sterk typesysteem
vergeleken. Dit met een nadruk op het typesysteem van de talen eerder dan de
talen zelf. Het gaat hier om twee functionele programmeertalen: Agda
\ref{agda}, een dependently typed programmeertaal, en Haskell \ref{haskell},
een taal met een Hindley-Milner \ref{hindmil} typesysteem.


\section{Typesysteem}

Een typesysteem is een verzameling van regels waaraan code in een bepaalde
programmeertaal moet voldoen. Er zijn vershillende redenen om zulke regels op
te leggen. De belangrijkste is type safety; dit betekent dat de computer nooit
onzinnige bewerkingen zal uitvoeren, wat een onzinnige bewerking juist is,
hangt af van het typesysteem. Gewoonlijk worden bewerking zoals optelling enkel
gedefinieerd op getallen, een voorbeeld van een onzinnige bewerking zou dan een
optelling kunnen zijn van een getal en een lijst. Een andere vorm van
bescherming die onder type safety valt is memory safety. Een belangrijke vorm
hiervan is bescherming tegen buffer overflows, deze hebben vaak security
vulnerabilities tot gevolg maar als het typesysteem over genoeg informatie
beschikt, kan gegarandeerd worden dat ze niet voorkomen.
