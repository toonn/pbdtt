\chapter{Besluit}
\label{besluit}

Dependent types zijn expressief en kunnen vele eigenschappen die we anders
ongezegd laten, duidelijk en correct formaliseren zonder veel moeite. De
afweging tussen moeite en statische garanties verdwijnt zeker niet maar is well
redelijk flexibel te bepalen. Gewoon met vectoren werken in plaats van lijsten
kost eigenlijk geen moeite en kan wel al mooie eigenschappen opleveren.
Uiteindelijk moet de afweging voorlopig nog vaak gemaakt worden op basis van
ervaring. Dependent types zijn nog niet zolang terug te vinden in \emph{echte}
programmeertalen en er zijn daardoor nog geen regels die we kunnen volgen over
hoeveel statische verificatie het grootste voordeel biedt tegenover de kleinste
inspanning. 

Wat we ook gezien hebben is dat, ook al zijn talen met dependent types nog niet
klaar voor algemeen gebruik, de ideeën die we erin opdoen gedeeltelijk
toepasbaar zijn in een taal zoals Haskell. Leren programmeren met dependent
types is dus geen tijdverspilling. Het heeft een dieper inzicht in het gebruik
van types in andere talen tot gevolg. En de technieken kunnen af en toe ook
overgezet worden naar die talen. 

De gevalstudies hebben laten zien dat bepaalde concepten helemaal niet moeilijk
te formaliseren zijn met dependent types. En dat door redelijk eenvoudige
formalisaties toch redelijk wat fouten kunnen worden voorkomen. We hebben ook
gezien dat Haskell redelijk ver te drijven is naar het dependently typed
programmeren, ook al wordt het steeds lastiger hoe dichter we bij dependent
types proberen komen. Hoewel ontwikkelingen zoals het singleton package voor
verbetering kunnen zorgen. Wat misschien nog de belangrijkste conclusie uit de
gevalstudies is, is dat als we een kleine toegeving doen op de precisie van de
types, we af en toe een grote vermindering van inspanning kunnen krijgen. En,
zolang de meeste eigenschappen nog in het typesysteem uitgedrukt zijn, kunnen
we die uit ons hoofd zetten en nadenken over belangrijkere problemen.

Alle code die besproken is, is beschikbaar in de bijlagen alsook op het
internet \url{https://github.com/toonn/pbdtt}.
