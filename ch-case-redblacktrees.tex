\chapter{Case: Verified Red-Black Trees}
\label{case:rbtree}


\section{Inleiding}

Deze tweede gevalstudie gaat over een implementatie van red-black trees
\cite{rbtrees}. Red-black trees zijn een soort binaire zoekbomen, door één
extra bit aan informatie bij te houden voor elke knoop zijn ze bij benadering
gebalanceerd. Deze benaderende balans is goed genoeg om $\mathcal{O}(\log{}n)$
tijdscomplexiteit te garanderen voor het opzoeken van een element, een element
toe te voegen en een element te verwijderen. Zoekbomen worden vaak gebruikt om
meer abstracte datastructuren te implementeren bijvoorbeeld verzamelingen en
associatieve arrays. Vooral voor de geordende varianten van die datastructuren
kunnen zoekbomen voordeliger zijn voor de implementatie dan hashtabellen omdat
zoekbomen per definitie geordend zijn.

Chris Okasaki heeft een elegante implementatie van red-black trees
\cite{okasaki} in een functionele programmeertaal geschreven, in Haskell.
De implementaties in dit hoofdstuk zijn hierop gebaseerd.
Zijn implementatie is gebaseerd op een bondige formulering van de red-black
tree invarianten. Deze gaan als volgt:

\begin{description}
  \item[Invariant 1] Geen enkele rode knoop heeft een rode ouder.
  \item[Invariant 2] Elk pad van de wortel naar een blad bevat hetzelfde aantal
                     zwarte knopen.
\end{description}

De implementatie van Okasaki wordt hier herhaald zodat het eenvoudig is om te
zien wat van het oorspronkelijk algoritme komt en wat toegevoegd is om de twee
invarianten statisch te garanderen.

\inputhaskell[breaklines]{haskell-casestt/Okasaki.hs}

Omdat deze implementatie zo kort en eenvoudig is kan bijna iedereen deze code
waarschijnlijk begrijpen. Maar omdat datastructuren zo fundamenteel zijn aan
alles wat programma's doen, is het belangrijk dat ze correct zijn. Eén
datastructuur kan nog wel correct gehouden worden door goed na te denken over
aanpassingen en een mentaal model van de werking bij te houden maar vanaf dat
er een tiental datastructuren zijn is er bijna niemand die ervoor kan zorgen
dat die correct zijn en blijven. Gewoonlijk wordt er daarom een suite van tests
opgesteld die regelmatig uitgevoerd dienen te worden, iedere keer dat er een
fout gevonden wordt, worden er tests toegevoegd die die fout zouden detecteren
als ze ooit terugkomt. Dit is een bewezen methode maar omvat onmiskenbaar een
hoop extra werk: tests moeten geschreven worden en onderhouden omdat
bijvoorbeeld de interface tot het programma kan veranderen. Een zwakheid van
deze aanpak is dat de tests vaak niet alle mogelijke gevallen afdekken: we
moeten vertrouwen op ons vermogen om te redeneren over de code om speciale
gevallen te ontdekken waar een grote kans is dat er iets mis gaat. Om dit op te
lossen zijn er gerandomiseerde testprogramma's, zoals QuickCheck
\cite{quickcheck}, bedacht die het vinden van speciale gevallen automatiseren.
In de implementaties in dit hoofdstuk daarentegen maken we gebruik van de
expressiviteit van dependent types om de invarianten door het typesysteem te
laten nakijken en zo de code statisch te verifiëren.


\section{Red-Black Trees in Agda}

De code van Okasaki is heel elegant maar ze bevat heel weinig statische
garanties. Het \ihask{Tree} type bijvoorbeeld voorkomt helemaal geen misbruik:
\ihask{T R (T R E 1 E) 2 E} is een geldige boom, net zoals \ihask{T B (T B E 1
E) 2 E} terwijl beiden niet voldoen aan de invarianten van red-black trees.
Dit betekent dat we goed moeten opletten als we een functie schrijven die een
\ihask{Tree} moet teruggeven omdat we geen waarschuwing krijgen als we een
ongeldige boom proberen op te stellen. Om nog maar niet te vermelden hoe we met
ongeldige bomen als invoer moeten omspringen? De gemakkelijkste oplossing is om
er niets aan te doen, maar dan kan het programma crashen omdat we ergens in een
functie impliciet aannamen dat de invoer een geldige red-black tree zou zijn.
Wat mogelijk nog erger is dan een crash is geen crash, een ongeldige boom gaat
door het programma en elke functie waarmee die in aanraking komt kan er iets
aan verandert hebben, op het einde merken we dat de uitvoer niet correct is
en nu moeten we uitzoeken waar het precies mis gegaan is. De expressiviteit van
dependent types kan ons hierbij helpen, als we ervoor zorgen dat een ongeldige
boom niet opgesteld kan worden, kunnen we ook geen problemen hebben met
ongeldige bomen. Het \iagda{Tree} type in Agda is eigenlijk een exacte
definitie van wat een red-black tree is terwijl het \ihask{Tree} type van
Okasaki eender welke binaire boom met rood en zwart gekleurde knopen voorstelt.

\inputagda[firstline=16,lastline=33,breaklines]{agda-casestt/RedBlackTree.agda}

We beginnen met een aantal imports en de definities van implicatie en mixfix
\emph{if then else}, alles dat we hier zien wordt uitgelegd daar waar het voor
het eerst gebruikt wordt.

\inputagda[firstline=35,lastline=43,breaklines]{agda-casestt/RedBlackTree.agda}

We definiëren een geparametriseerde module, \iagda{RBTree}, met als parameter
een record van het type \iagda{StrictTotalOrder}. De \iagda{a} en \iagda{ℓ}
parameters zijn levels en zorgen ervoor dat de elementen van onze boom ook
elementen van een type in één van de grotere universa kunnen zijn. Door een
element van het type \iagda{StrictTotalOrder} te eisen, kunnen we er zeker van
zijn dat we de elementen kunnen ordenen en de zoekbomen dus nuttig kunnen
invullen. De twee volgende regels zorgen ervoor dat we in de rest van de code
in de module kunnen verwijzen naar dit type met een strikte totale orderelatie
als \iagda{A}. De pattern declaraties zorgen ervoor dat we met een compactere
notatie kunnen werken voor het resultaat van een vergelijking. Op deze manier
verbergen we een aantal bewijzen die we niet echt nodig hebben. Ten slotte
kiezen we een symbool om de vergelijkingsfunctie te verkorten. Het symbool
staat voor de relatie "kleiner dan of gelijk aan" maar het resultaat van de
functie is \iagda{LT} als het eerste argument voor het tweede komt, \iagda{EQ}
als de elementen gelijk zijn en \iagda{GT} als het eerste element na het tweede
komt.

Nu kunnen we net als Okasaki beginnen met definiëren wat een kleur is en wat een
red-black tree is.

\inputagda[firstline=46,lastline=59,breaklines]{agda-casestt/RedBlackTree.agda}

Het type \iagda{Color} is heel eenvoudig, de functie \iagda{_=ᶜ_} is niet meer
dan een vergelijking van twee kleuren op gelijkheid met een Booleaans
resultaat. We definiëren ook nog een synoniem voor natuurlijke getallen,
\iagda{Height}, wat het aantal zwarte knopen op een pad van de wortel naar een
willekeurig blad voorstelt. Het type \iagda{Tree} is ook nog redelijk eenvoudig
maar dwingt wel meteen de twee invarianten af. De constructor voor een rode
knoop, \iagda{R}, laat enkel kinderen toe met een zwarte wortel, op die manier
kan een rode knoop nooit een rode ouder hebben. Dit zorgt dat invariant 1
behouden blijft. Overigens moeten de twee kinderen dezelfde hoogte hebben om
invariant 2 te behouden. De constructor, \iagda{B}, eist eveneens dat de
kinderen een gelijke hoogte hebben zodat invariant 2 behouden blijft. Meer is
er niet nodig om een type op te stellen dat enkel correcte red-black trees kan
voorstellen.

Het grootste voordeel van dit preciezere type is tegelijk ook een nadeel:
algoritmes verbreken vaak tijdelijk een of meerdere invarianten omdat het soms
eenvoudiger is om iets kapot te maken en daarna in stappen te herstellen dan om
alles in één keer goed te doen. Dit is vooral het geval wanneer een verandering
een andere verandering tot gevolg kan hebben met een domino-effect. Okasaki
maakt gebruik van deze techniek in zijn \ihask{ins} functie: \ihask{ins} zou
een ongeldige boom kunnen teruggeven en daarom zijn de oproepen naar
\ihask{ins} als het ware afgeschermd door \ihask{makeBlack} of \ihask{balance}.
De \ihask{ins} functie kan een boom teruggeven met een rode wortel en één rood
kind, dit noemen we ook wel een infrarode boom. Een infrarode boom kan altijd
hersteld worden door de wortel zwart te kleuren maar dit vergroot de hoogte met
één. Tijdens het toevoegen van een element vermijden we liever veranderingen
van hoogte omdat als één kind hoger wordt dan het andere, we de boom moeten
herbalanceren. De recursieve oproepen naar \ihask{ins} zijn daarom afgeschermd
door oproepen naar \ihask{balance}. Het is tevens het algoritme voor
herbalancering waar de oplossing van Okasaki zijn elegantie vandaan haalt. Als
het resultaat van een recursieve oproep naar \ihask{ins} een infrarode boom is,
dan moet de wortel van het kind waarin we de waarde hebben toegevoegd, rood
zijn geweest en dat kan alleen als de ouder van dat kind zwart was. Dit zijn nu
net de voorwaarden voor de \ihask{balance} functie van Okasaki. Omdat we zulk
een tijdelijke schending van de invarianten niet kunnen voorstellen in ons
\iagda{Tree} type, hebben we een nieuw type nodig dat infrarode bomen kan
voorstellen.

\inputagda[firstline=61,lastline=63,breaklines]{agda-casestt/RedBlackTree.agda}

Omdat we nu twee verschillende types hebben voor red-black trees en infrarode
bomen kunnen we de \iagda{balance} functie niet implementeren zoals Okasaki.
Soms zou het type van het linker argument infrarood moeten zijn en het rechts
red-black en soms omgekeerd. We kunnen dit gedrag wel vatten in een nieuw type.

\inputagda[firstline=65,lastline=67,breaklines]{agda-casestt/RedBlackTree.agda}

Dankzij de flexibele mixfix syntax van Agda kunnen we nu deze toepassing uit
Haskell, \ihask{balance B infra b c}, als volgt schrijven in Agda:
\iagda{balance (infra ◂ b ◂ c)}. We hebben ook geen nood meer aan het kleur
argument, waarom komt later aan bod.
We komen nog één type te kort, sommige functies zoals \iagda{ins} kunnen zowel
een geldige als een ongeldige boom teruggeven. Omdat deze voorgesteld worden
door twee aparte types hebben we een type nodig dat werkt als een disjuncte
vereniging van die twee types.

\inputagda[firstline=69,lastline=71,breaklines]{agda-casestt/RedBlackTree.agda}

Meer precisie kost meer werk, wat bij Okasaki één type was, zijn er nu vier.
Hoe preciezer we willen zijn hoe meer we expliciet moeten maken: dit zal ook
duidelijk worden in de implementaties van de functies. De eerste functie die we
bekijken is \iagda{balance}.

\inputagda[firstline=75,lastline=79,breaklines]{agda-casestt/RedBlackTree.agda}

De implementatie gelijkt sterk op die van Okasaki maar er zijn subtiele
verschillen. Het kleur argument is niet langer nodig, Okasaki gebruikt dit
eigenlijk alleen in het geval dat hij niets doet met de invoer om de argumenten
terug samen te stellen tot een geldige boom met dezelfde kleur wortel. Omdat
onze functie enkel bomen aanvaard die gebalanceerd \emph{moeten} worden en een
gebalanceerde boom altijd rood is, weten we op voorhand wat de kleur van de
boom is die we teruggeven. Dit kunnen we alleen maar doen omdat ons type nu
strikt genoeg is om geldige bomen uit te sluiten van \iagda{balance}. Dit
betekent wel dat we op de plaats waar we \iagda{balance} willen gebruiken,
zeker moeten zijn dat we een ongeldige boom hebben. Om die beslissing te kunnen
maken moeten we vaak meer gevallen van mekaar scheiden maar dit heeft ook
voordelen. In definities zoals die van Okasaki worden sommige functies soms
uitgevoerd zonder dat ze iets doen, als er dan een bug optreedt, moeten ook
deze functies nagegaan worden omdat we niet op voorhand weten of de functies de
waarde inderdaad onveranderd laten.

De functies die Okasaki met een locale definitie implementeert in
\ihask{insert} zijn meer veranderd. Ze zijn niet meer lokaal gedefinieerd omdat
de \iagda{where} expressie uit Agda maar geldt voor één vergelijking. De
eenvoudigste van die functies is \iagda{blacken}, die hetzelfde doet als
\ihask{makeBlack}.

\inputagda[firstline=81,lastline=87,breaklines]{agda-casestt/RedBlackTree.agda}

Het enige dat \iagda{blacken} doet, is de wortel van een boom zwart kleuren.
Dit wordt bemoeilijkt omdat we dit zowel voor een geldige als voor een
ongeldige boom moeten doen, vandaar het type \iagda{Treeish} voor het argument.
Omdat we een nieuwe knoop teruggeven met een andere constructor moeten we de
invoer volledig uit elkaar halen en terug samenstellen in het resultaat, daarom
zijn er zoveel meer vergelijkingen. Dit is de eerste keer dat we zien dat het
type van het resultaat een functie oproep is. In dit geval hangt het af van de
kleur van de wortel van de invoer of de boom hoger wordt of niet: als een rode
wortel zwart wordt gekleurd verhoogt de boom. Daarom gebruiken we
\iagda{if_then_else_} in het type en omdat we met dependent types werken is dit
geen probleem: het resultaat van een functie kan een type zijn en dit type
kunnen we ook meteen gebruiken als type.

Omdat \iagda{ins} niet meer lokaal gedefinieerd is moeten we het argument dat
we gaan toevoegen expliciet maken. De functie \iagda{ins} geeft een red-black
tree terug als de invoer een zwarte boom was en geeft een red-black tree of een
infrarode boom terug, in de vorm van een \iagda{Treeish}, als de invoer een
rode boom was. De hoogte van het resultaat is dezelfde als die van de invoer,
wat enkel verzekerd kan worden omdat we een infrarode boom kunnen teruggeven:
het is het zwart maken van de wortel dat de hoogte kan verhogen. De kleur van
de boom die we teruggeven, kunnen we niet zomaar bepalen daarom is de kleur
existentieel gekwantificeerd met een dependent pair, een koppel waarvan het
type van het tweede element kan afhangen van de waarde van het eerste. Dit
betekent wel dat we altijd een koppel van een kleur en een boom moeten
teruggeven.

\inputagda[firstline=89, lastline=111,
           breaklines]{agda-casestt/RedBlackTree.agda}

Wat meteen opvalt is dat de definitie van \iagda{ins} veel langer is dan in de
implementatie van Okasaki. Dit is vooral omdat we veel preciezer moeten zijn.
Wat voor boom we teruggeven en of we moeten balanceren of niet, maakt veel uit.
De \iagda{with} expressie laat ons toe om de tussentijdse resultaten te
inspecteren die we nodig hebben om te beslissen wat voor boom we teruggeven en
of we moeten herbalanceren. De code leest als volgt, bijvoorbeeld voor het
derde deel waar we \iagda{ins} toepassen op een boom met een zwarte wortel,
\iagda{(B _ b _)}. Als het element dat we wensen toe te voegen, \iagda{a},
kleiner, \iagda{LT}, is dan het element in de wortel, \iagda{b}, en de
toevoeging van het element aan het linkse kind, \iagda{ins a l}, heeft als
resultaat \iagda{c , RB t} dan is de boom die we teruggeven zwart, \iagda{B , B
t b r}. Als daarentegen het resultaat van \iagda{ins a l}, \iagda{.R , IR t}
is, dan is het resultaat sowieso een rode boom en moeten we herbalanceren,
\iagda{R , balance (t ◂ b ◂ r)}. De dot patterns in Agda, zoals \iagda{.R},
betekenen dat die waarde de enige mogelijke waarde is wat door Agda nagegaan
wordt.

De \iagda{insert} functie die we nu kunnen definiëren is preciezer dan die van
Okasaki, maar is niet zo precies mogelijk: de mogelijk hoogtes van het
resultaat zijn beperkt maar de hoogte is niet volledig bepaald. In de import
declaratie voor de disjuncte vereniging, \iagda{Sum}, hebben we de constructors
een naam gegeven die aangeeft of een boom van gelijke hoogte is als de invoer,
\iagda{h+0} of één hoger geworden is \iagda{h+1} dit maakt het resultaat iets
makkelijker leesbaar. Dat het resultaat een disjuncte vereniging is, is een
toegeving. Het probleem met preciezer zijn, is dat het moeilijk op voorhand te
bepalen is of een boom hoger zal worden na het toevoegen van een element. Een
functie die dit bepaalt, is wel gemakkelijk te schrijven en kunnen we gebruiken
in het type maar Agda kan niet redeneren over zo'n complexe functie en in
combinatie met de totaliteitsvoorwaarde voor functies in Agda zorgt dit voor
problemen waar een invoer, die eigenlijk niet kan gegeven worden, ervoor zorgt
dat we een uitvoer zouden moeten geven die we niet kunnen opstellen. Een
oplossing hiervoor kan waarschijnlijk gevonden worden met een bewijs dat
duidelijk maakt voor Agda wat wel en niet kan voorkomen. De aanzienlijke
moeilijkheid om zo een bewijs op te stellen tegenover de extra precisie die we
verkrijgen voor de hoogte is in dit geval niet de moeite waard gevonden. Hier
hebben we dus gekozen voor eenvoud over precisie.

\inputagda[firstline=113, lastline=120,
           breaklines]{agda-casestt/RedBlackTree.agda}

De \iagda{insert} functie neemt een waarde en een red-black tree en geeft een
red-black tree terug van dezelfde hoogte of één hoger. De gevallen voor een
rode en een zwarte boom moeten we opsplitsen omdat \iagda{ins} respectievelijk
een \iagda{Tree} of een \iagda{Treeish} teruggeeft.

Nu rest er ons enkel nog de definities voor verzamelingen op basis van de bomen
te geven. Ze zijn niet even handig geformuleerd als die van Okasaki maar de
focus van deze gevalstudie ligt hier ook niet op.

\inputagda[firstline=123, lastline=139,
           breaklines]{agda-casestt/RedBlackTree.agda}

Ten slotte is er nog een voorbeeld van hoe we met deze bomen kunnen werken, het
interessantste hieraan zijn de bewijsjes vanonder die aantonen dat een bepaalde
boom gelijk is aan een toevoeging van een element aan een andere boom.

\inputagda[firstline=146, lastline=172,
           breaklines]{agda-casestt/RedBlackTree.agda}


\section{Red-Black Trees in Haskell}

In Haskell beginnen we opnieuw met de declaratie van de LANGUAGE pragma's die
we nodig hebben en de definitie van natuurlijke getallen. Deze keer gebruiken
we geen singleton types dus blijft het eenvoudiger. Ook de kleuren zijn nog
steeds eenvoudig.

\inputhaskell[firstline=16, lastline=21,
              breaklines]{haskell-casestt/RedBlackTree.hs}

Het \ihask{Tree} type is vrij gelijkaardig aan het \iagda{Tree} type uit de
Agda code.

\inputhaskell[firstline=23, lastline=26,
              breaklines]{haskell-casestt/RedBlackTree.hs}

Het grootste verschil, buiten de namen van de constructors, is dat het
\ihask{Tree} type een extra parameter heeft voor het polymorfisme. In Agda
konden we door de parameter van de module als het ware overal aannemen dat het
type van de elementen gekend was. Haskell heeft geen geparametriseerde modules
dus maken we al onze types polymorf en voegen we constraints toe aan de
functies die vergelijkingen moeten kunnen doen.

Wat we nog niet zijn tegengekomen is het implementeren van een typeclass voor
een GADT, dit kan wel eens verrassend moeilijk zijn. Daarom tonen we hier een
voorbeeld voor de \ihask{Eq} typeclass.

\inputhaskell[firstline=28, lastline=48,
              breaklines]{haskell-casestt/RedBlackTree.hs}

Omdat de kleuren van de kinderen van een zwarte knoop onbepaald zijn moeten we
veel meer pattern matches doen omdat Haskell anders niet genoeg kan afleiden
over de hoogtes van de kinderen. Het enige nut van deze typeclass is om te
kunnen testen of een boom die we uit \ihask{insert} krijgen gelijk is aan een
andere boom wat alleen tijdens de uitvoering kan bepaald worden.

Omdat het \ihask{Tree} type opnieuw specifieker is dan bij Okasaki komen we
hetzelfde probleem tegen en hebben we een aantal extra types nodig.

\inputhaskell[firstline=51, lastline=61,
              breaklines]{haskell-casestt/RedBlackTree.hs}

Op een paar verschillen na, zoals dit zijn GADTs waar we in Agda inductive
families gebruiken. Hier werken we met polymorfisme en Haskell laat geen mixfix
notatie toe, gelijken deze types sterk op de types uit de Agda implementatie.

De \ihask{balance} functie is opnieuw zeer gelijkaardig.

\inputhaskell[firstline=65, lastline=69,
              breaklines]{haskell-casestt/RedBlackTree.hs}

De functie \ihask{blacken} heeft een iets minder precies type. Functies op het
niveau van types zijn in Haskell veel meer beperkt, net omdat er een verschil
is tussen waarden en types, terwijl in Agda alle types waarden zijn. De
TypeFamilies extensie \cite{typefam} van GHC biedt iets wat lijkt op een
functie op het typeniveau. Hier kiezen we opnieuw voor een eenvoudigere
oplossing door het type minder precies te maken.

\inputhaskell[firstline=71, lastline=76,
              breaklines]{haskell-casestt/RedBlackTree.hs}

Het type dat we teruggeven is nu een disjuncte vereniging, we moeten dus
expliciet aangeven of het resultaat dezelfde hoogte heeft of één hoger geworden
is.

De \ihask{ins} functie heeft ook een iets minder streng type, we geven nu
altijd een \ihask{Treeish} terug in plaats van af en toe een \ihask{Tree}. In
dit geval vervangen we het dependent pair uit Agda door een disjuncte
vereniging. Dit gaat alleen als het type waarover het dependent pair
existentieel kwantificeert, exact twee elementen heeft. In deze functie moeten
we ook de volgorde van elementen kunnen bepalen, daarom leggen we een
\ihask{Ord} constraint op, op het type van de elementen.

\inputhaskell[firstline=78, lastline=97,
              breaklines]{haskell-casestt/RedBlackTree.hs}

De functie maakt nu gebruik van pattern guards \cite{patguard} omdat
\ihask{ins} altijd een \ihask{Treeish} teruggeeft en we aan de boom of
infrarode boom moeten geraken die daar in zit. Dit is niet gewoon een kwestie
van pattern matchen en het is enkel gelukt met pattern guards, \ihask{case}
expressies zouden dus ook moeten werken. Dat het type dat \ihask{ins}
teruggeeft altijd \ihask{Treeish} is, zou problemen geven in Agda. Omdat Agda
een totale programmeertaal is, zouden we als we willen pattern matchen op een
functie die een \ihask{Treeish} teruggeeft, een match moeten voorzien voor elk
geval dat mogelijk is volgens het type. Concreet zou dit willen zeggen voor
\ihask{ins} dat we een match moeten voorzien voor een infrarode boom, ook al
geven we als invoer een zwarte boom, wat eigenlijk niet kan voorkomen. In het
kort komt het erop neer dat we in Agda een functie zouden moeten definiëren om
het geval te dekken dat niet kan voorkomen. Dit heeft natuurlijk niet veel zin
en het is logischer om het type preciezer te maken. In Haskell geeft dit geen
probleem omdat we partiële functies mogen definiëren. Omdat er niet
gecontroleerd wordt op totaliteit kan het wel zijn dat we ergens een geval over
het hoofd zien dat toch kan voorkomen dus moeten we beter opletten.

De \ihask{insert} functie is terug eenvoudiger omdat \ihask{ins} nu altijd een
\ihask{Treeish} teruggeeft. Het type is ongeveer dat uit Agda, op polymorfisme
en de \ihask{Ord} constraint na. En we gebruiken weer een pattern guard om het
tussentijds resultaat uit te pakken vooraleer we het aan \ihask{blacken} kunnen
geven.

\inputhaskell[firstline=99, lastline=102,
              breaklines]{haskell-casestt/RedBlackTree.hs}

De functies voor verzamelingen hebben ongeveer hetzelfde nadeel als in Agda.

\inputhaskell[firstline=105, lastline=121,
              breaklines]{haskell-casestt/RedBlackTree.hs}

En tenslotte kunnen we opnieuw een voorbeeld geven van het gebruik, jammer
genoeg zijn de \emph{bewijzen} deze keer alleen na te gaan door ze uit te
voeren.

\inputhaskell[firstline=126, lastline=149,
              breaklines]{haskell-casestt/RedBlackTree.hs}


\section{Besluit}

Uit deze gevalstudie is duidelijk dat we altijd kunnen kiezen hoever we de
verificatie doordrijven. De belangrijke invarianten over de volgorde van
element in een zoekboom zijn hier niet opgenomen in het typesysteem. Dit kan
verschillende motivaties hebben, misschien vinden we het te eenvoudig om de
vergelijkingen voor volgorde juist te doen en hebben we geen behoefte aan de
zekerheid die we zouden krijgen van een statische verificatie. Ook hebben we
gezien dat als we af en toe een type een beetje verzwakken, dat de code dan heel
eenvoudig kan blijven. In deze gevalstudie was de code in Haskell helemaal niet
moeilijker te schrijven omdat we gebruik gemaakt hebben van het feit dat
Haskell partiële functies toelaat. Verder zijn we in onze opzet geslaagd om
ervoor te zorgen dat bepaalde functies enkel geldige bomen terug kunnen geven.
In een interface van een library met datastructuren kan dit handig van pas
komen omdat we kunnen garanderen aan de gebruiker dat wat hij krijgt altijd een
geldige datastructuur is.

Wat betreft de implementatie van zoekbomen is deze gevalstudie eerder een
voorbeeld van een kleine stap richting formalisatie. Af en toe heeft iemand een
heel goed idee en in dit geval is er bijvoorbeeld een artikel van Conor
McBride \cite{hkynio} met de titel: "How to Keep Your Neighbours in Order."
In dit artikel formuleert McBride een elegante en heel erg algemene
voorstelling voor geördende bomen en andere datastructuren.

De volledige code voor dit hoofdstuk is terug te vinden in appendices
\ref{app:case-rbtree-agda} en \ref{app:case-rbtree-haskell}.
In het volgende hoofdstuk zien we nog een besluit over het geheel.
