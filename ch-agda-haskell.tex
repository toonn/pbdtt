\chapter{Programmeertalen}
\label{ch:agda-haskell}

In dit hoofdstuk beschrijven we de programmeertalen die gebruikt zijn in de
gevalstudies. Er moet ergens een grens getrokken worden die bepaald wat wel en
niet uitgelegd wordt, deze thesis veronderstelt dat de lezer vertrouwd is met
getypeerd functioneel programmeren.
Omdat Agda minder bekend is, is de uitleg hierover uitgebreider dan die over
Haskell. Het enige dat we voor Haskell moeten uitleggen zijn namelijk de
extensies van GHC \ref{ghc} die we gebruiken.


\section{Agda}

Agda is een dependently typed programmeertaal. Talen met dependent types zijn
dankzij de Curry-Howard correspondence ook bruikbaar als bewijsassistent. In
vele andere talen met dependent types ligt de nadruk ook eerder op het gebruik
als bewijsassistent dan wel als programmeertaal.

\subsection{Dependent Types}

Het belangrijkste verschil tussen Agda en andere functionele programmeertalen
is het typesysteem. Een dependent type is een type dat afhangt van een waarde.
De voorgaande zin legt eigenlijk alles uit dat belangrijk is aan dependent
types maar iemand die nog niet vertrouwd is met dependent types zal hier weinig
van opsteken. Vandaar overlopen we een aantal voorbeelden in Agda: eerst om de
syntax te verduidelijken, daarna om te illustreren wat dependent types nu
eigenlijk zijn.

De definitie van nieuwe types gebeurt gelijkaardig als in andere getypeerde
functionele programmeertalen. Een \iagda{data} sleutelwoord wordt gevolgd door
de naam van het type, een dubbel punt, het type van het type, dan het
sleutelwoord \iagda{where} en op de volgende regels de constructors met hun
type. Nemen we als voorbeeld een definitie voor natuurlijke getallen:

%----------------------------------------%
\begin{minted}[fontsize=\small]{agda}
  data Nat : Set where
    zero : Nat
    suc  : Nat → Nat
\end{minted}

Dit geeft ons een unaire voorstelling van de natuurlijke getallen, het getal 2
stellen we bijvoorbeeld voor als volgt: \iagda{suc (suc zero)}. Een verschil
met andere talen is dat we het type van het type moeten opgeven.  In
dependently typed talen is de scheiding tussen types en waarden fundamenteel
opgeheven. Het type van de meeste eenvoudige types is in Agda \iagda{Set}, wat
eigenlijk \iagda{Set₀} is. Het type van \iagda{Set₀} is \iagda{Set₁} en deze
hiërarchie gaat in theorie oneindig ver door. Haskell heeft gelijkaardige
concepten maar daar gaat de hiërarchie niet erg ver door. De hiërarchie gaat
als volgt in Haskell, met de Engelse termen omdat ze niet allemaal een goede
vertaling hebben: een value heeft een type, een type heeft een kind, een kind
heeft een sort en hier stopt de hiërarchie. Het dichtste equivalent van
\iagda{Set} is in Haskell het kind \ihask{*}, \ihask{*} heeft nog sort
\ihask{BOX} maar \ihask{BOX} heeft zelf sort \ihask{BOX}. Merk op dat
\ihask{BOX} enkel en alleen een intellectueel concept is, het is niet uit te
drukken in Haskell zelf.  Het tweede verschil met een type declaratie in een
taal zoals Haskell is dat we het type van elke constructor moeten opgeven, voor
\iagda{Nat} is dit heel eenvoudig. In het volgende voorbeeld wordt duidelijker
waarom we deze types moeten specificeren.

Dit voorbeeld wordt vaak gebruikt om het concept van dependent types te
illustreren. Deze code komt uit de Agda Standard Library \ref{agda:stdlib} maar
is licht aangepast om het voorbeeld eenvoudiger te maken.

%----------------------------------------%
\begin{minted}[fontsize=\small]{agda}
  data Vec (A : Set) : ℕ → Set where
    []  : Vec A zero
    _∷_ : ∀ {n} (x : A) (xs : Vec A n) → Vec A (suc n)

  head : ∀ {n} {A : Set} → Vec A (suc n) → A
  head (x ∷ xs) = x
\end{minted}

\iagda{Vec} is het type voor lijsten met een vaste lengte, vanaf nu noemen we
dit vectors. Hier zien we twee nieuwe dingen, het type \iagda{Vec} verwacht nog
twee argumenten. Het eerste is de parameter voor de dubbel punt, \iagda{A} met
als type \iagda{Set} dus \iagda{A} is een eenvoudig type. Het tweede is de
index van het type \iagda{ℕ}, dit is de naam van het type voor natuurlijke
getallen uit de standard library, het verschil tussen een index en een
parameter is dat een index voor elke constructor kan variëren terwijl een
parameter voor alle constructors hetzelfde is. Een type met indices noemen we
ook wel een inductieve familie \ref{indfam}. De lege vector, \iagda{[]},
heeft lengte nul, het type is \iagda{Vec A zero}: het is dus een vector van
elementen van type \iagda{A} met lengte \iagda{zero}. De constructor die
langere vectoren maakt, verwacht een element, \iagda{x} van type \iagda{A}, een
vector met elementen van hetzelfde type \iagda{A} en een bepaalde lengte
\iagda{n}, namelijk \iagda{xs} met als type \iagda{Vec A n}, en geeft een
vector terug waarvan de lengte één groter is, \iagda{Vec A (suc n)}. De
\iagda{head} functie, die het eerste element van een vector terug geeft kan dan
eisen dat ze enkel werkt op vectoren met een lengte groter dan nul, vectoren
met type \iagda{Vec A (suc n)}. Als de lengte niet in het type opgenomen is,
kan een functie er ook geen voorwaarden aan opleggen. Zo moet de \ihask{head}
functie voor gewone lijsten in Haskell een fout opwerpen wanneer ze opgeroepen
wordt met een lege lijst als argument.  Dit is een heel eenvoudig voorbeeld en
hieruit blijkt niet welke van de twee een betere manier is om lijsten voor te
stellen.  Het laat wel zien wat het betekent voor een type om af te hangen van
een waarde. Wat we ook zien, zowel in het type van \iagda{_∷_} als \iagda{head}
is \iagda{∀ {n}}, de universele kwantor zorgt dat \iagda{n} eender wat kan
zijn zolang het voldoet aan de eisen in de rest van het type. De accolades rond
\iagda{n} en in het type van \iagda{head} ook rond het type van het volgende
argument, \iagda{{A : Set}}, duidt aan dat deze argumenten impliciet zijn, ze
worden afgeleid uit de context, in dit geval bijvoorbeeld uit het type van het
vector argument.

Een meer uitgebreid voorbeeld uit een artikel dat het nut van dependent types
goed uitlegt \ref{TPoP}, laat zien dat dependent types zeer expressief zijn.
Het type \iagda{RA} laat toe om een relationele algebra expressie op te stellen
die correct is door constructie. Het artikel beargumenteert dat het in Haskell
niet mogelijk is om een interface naar een database te voorzien die evenveel
statische garanties biedt en even volledig - joins zijn zonder dependent types
moeilijk te typeren - is zonder gebruik te maken van preprocessing of
experimentele features. Dit betekent niet dat er geen libraries bestaan voor
Haskell die bepaalde eigenschappen statisch of dynamisch op leggen, maar wel
dat die beperkter zijn en minder statisch kunnen zijn. Statische correctheid is
handiger omdat ze geen uitvoerige testen noodzakelijk maakt.

%----------------------------------------%
\begin{minted}[fontsize=\small]{agda}
  data RA : Schema → Set where
    Read    : ∀ {s} → Handle s → RA s
    Union   : ∀ {s} → RA s → RA s → RA s
    Diff    : ∀ {s} → RA s → RA s → RA s
    Product : ∀ {s s'} → {_ : So (disjoint s s')} → RA s → RA s'
              → RA (append s s')
    Project : ∀ {s} → (s' : Schema) → {_ : So (sub s' s)} → RA s → RA s'
    Select  : ∀ {s} → Expr s BOOL → RA s → RA s
\end{minted}

Het type \iagda{Schema} stelt een schema van een relatie voor, door dit als
index op te nemen, kunnen we bepaalde eisen stellen aan de schema's waarop een
relationele expressie werkt. De \iagda{Read} constructor gelijkt een beetje op
een lege lijst omdat die aan de basis van elke relationele expressie moet
liggen, voor alle andere constructors voor een relationele expressie hebben we
al een relationele expressie nodig. \iagda{Read} bevat de informatie die
aangeeft in welke database de relatie te vinden is in een \iagda{Handle}. Zo'n
\iagda{Handle} heeft tevens een schema als index en kunnen we enkel opstellen
als de database inderdaad een relatie heeft die aan het schema voldoet. Voor de
unie van twee relaties, \iagda{Union}, en het verschil tussen twee relaties,
\iagda{Diff}, eisen we dat de schemas overeenkomen: een relatie kan maar een
unie zijn van twee relaties als alle attributen overeenkomen, eigenlijk moeten
de schema's niet hetzelfde zijn maar mogen ze permutaties van elkaar zijn maar
dit kan opgelost worden door de implementatie van \iagda{Schema}. De
constructor voor het cartesisch product, \iagda{Product}, heeft een impliciet
argument van het type \iagda{So (disjoint s s')}, de underscore duidt aan dat
we het argument niet gebruiken in de rest van het type maar moet er staan omdat
we een type van een impliciet argument niet zonder een variabele kunnen
opgeven. Het type \iagda{So (disjoint s s')} zorgt ervoor dat de schemas
\iagda{s} en \iagda{s'} disjunct zijn omdat een relatie geen twee attributen
met dezelfde naam kan hebben en een cartesisch product geen join is. Verder
zien we dat \iagda{Product} een relationele expressie opstelt waarvan het
resulterende schema de combinatie van de disjuncte schema's is. Voor de
projectie, \iagda{Project}, eisen we dat de attributen die we uit een relatie
willen projecteren inderdaad aanwezig zijn in de relatie. Het belangrijkste
kenmerk van de \iagda{Select} constructor kunnen we helaas niet uitleggen
zonder meer van het artikel over te nemen maar is niet belangrijk in de rest
van dit werk.

\subsection{Syntax}

Wat ook opmerkelijk is aan Agda, is de vrijheid die er is door de flexibele
syntax. Agda legt geen regels op in verband met hoofdletters voor types en
constructors en kleine letters voor functies, dit mes snijdt natuurlijk aan
twee kanten omdat types en functies minder gemakkelijk uit elkaar te houden
zijn. Wat waarschijnlijk de belangrijkste reden is om zulke regels niet af te
dwingen is dat er eigenlijk geen verschil is tussen waarden en types, een
functie is ook een waarde, het is dus ook niet logisch om artificieel een
verschil op te leggen in de schrijfwijze. Wat naamgeving betreft is nog een
belangrijk kenmerk dat alle unicode karakters toegelaten zijn. Hier wordt in de
standard library ook veel gebruik van gemaakt. Praktisch betekent dit dus dat
er goede ondersteuning moet zijn van de tekstverwerker en het lettertype waarin
je Agda code schrijft.

Een belangrijker kenmerk van de syntax in Agda is dat types, constructors en
functies gedefinieerd kunnen worden met mixfix notatie. Voor vectoren hebben we
al een infix constructor gezien, een tweede voorbeeld van zulke notatie is dit:

%----------------------------------------%
\begin{minted}[fontsize=\small]{agda}
  if_then_else_ : {A : Set} → Bool → A → A → A
  if true then x else y = x
  if false then x else y = y
\end{minted}

Deze functie kunnen we nu op twee manieren gebruiken. Met prefix notatie als we
de naam met underscores overnemen als volgt: \iagda{if_then_else_ true x y}. Of
zoals in de definitie in de mixfix vorm. Soms maakt dit de code gemakkelijker
leesbaar. Wat wel noodzakelijk wordt door zo'n flexibele notatie is het
veelvuldig gebruik van spaties, voor \iagda{if_then_else_} maakt dit weinig uit
want daar zijn de spaties logisch maar stel bijvoorbeeld dat dit een functie
is: \iagda{[_]}, dan moeten er spaties rond het argument, \iagda{[ x ]}, wat in
het begin vreemd aanvoelt maar nodig is omdat \iagda{[x]} een geldige naam zou
kunnen zijn voor een functie.


\section{Haskell}

Omdat we redelijk veel vergen van het typesysteem, komen we niet toe met
standaard Haskell zoals bijvoorbeeld geïmplementeerd in GHC maar hebben we een
aantal extensies nodig van GHC.

\subsection{Waarden op typeniveau}

Omdat we waarden nodig hebben op het niveau van types, wat dus eigenlijk types
zijn, en die types zelf een type moeten hebben, wat in Haskell dus een kind is,
maken we gebruik van de extensie DataKinds. Deze extensie laat ons toe om zelf
kinds te definiëren met bijhorende types door onze gewone types te promoveren
tot kinds en hun constructors te promoveren tot types. Een voorbeeld maakt dit
veel duidelijker:

%----------------------------------------%
\begin{minted}[fontsize=\small]{haskell}
  data Nat = Z | S Nat

  two :: Nat
  two = S (S Z)

  type_level_two :: S (S Z)
\end{minted}

We definiëren gewoon een type voor natuurlijke getallen en door de DataKinds
extensie wordt dit type een kind en de constructors types. Het kind \ihask{Nat}
is dus het kind van twee types namelijk \ihask{Z} en \ihask{S Nat}. De
\ihask{Nat} in \ihask{S Nat} is opnieuw het kind Nat niet het type. Omdat
dezelfde namen voor verschillende concepten af en toe verwarrend kunnen zijn en
het niet onmogelijk is dat we al een type \ihask{Z} gedefinieerd hadden, worden
types en constructors ook altijd gepromoveerd tot kinds en types die beginnen
met een apostrophe. Het type \ihask{Nat} wordt bijvoorbeeld altijd gepromoveerd
tot de kind \ihask{'Nat} en de constructor \ihask{Z} tot het type \ihask{'Z}.
Met deze extensie kunnen we dus waarden voorstellen op typeniveau waar de
extensie nog tekortschiet is dat de waarde \ihask{Z} en het type \ihask{Z}
volledig los van elkaar staan, buiten de naam. Er is ook geen waarde van het
type \ihask{Z}. In één van de volgende hoofdstukken gebruiken we een techniek
waarmee we de waarden op waardeniveau en typeniveau ongeveer kunnen verbinden.

\subsection{Inductieve families}

\subsection{multiparam, flexinst, polykinds}


